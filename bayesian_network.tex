\documentclass[12pt]{article}
\usepackage{geometry}                % See geometry.pdf to learn the layout options. There are lots.
\geometry{letterpaper}                   % ... or a4paper or a5paper or ... 
%\geometry{landscape}                % Activate for for rotated page geometry
\usepackage[parfill]{parskip}    % Activate to begin paragraphs with an empty line rather than an indent
\usepackage{amssymb}
\usepackage{amsmath}
\usepackage{hyperref}

\title{Conditional Independence in Bayesian Network}
\author{Kai Zhang \\
\href{mailto:kz298@cornell.edu}{kz298@cornell.edu}
}
\date{\today}                                           % Activate to display a given date or no date

\begin{document}
\maketitle

\section*{Problem}
A Bayesian network is a DAG $G=(V, E)$. Suppose a subset $S\subset V, |\bar{S}|\geq 2$ is observed, our problem is that  for any given $v_1,v_2\in \bar{S},v_1\neq v_2$, check if $v_1\perp v_2\mid S$ holds.

\section*{Solution}
First we define a subgraph graph $G_{\bar{S}}\subset G$ consisting of vertices in $\bar{S}$. Then we define 5 sets of vertices as below.
\begin{align*}
&\Phi_1=\big\{v\in \bar{S}: v_1\rightsquigarrow v\text{ in } G_{\bar{S}}\big\}\\
&\Phi_2=\big\{v\in \bar{S}: v_2\rightsquigarrow v\text{ in } G_{\bar{S}}\big\} \\
&\Phi_3=\big\{v\in \bar{S}: v\rightsquigarrow v_1\text{ in } G_{\bar{S}}\big\}\\
&\Phi_4=\big\{v\in \bar{S}: v\rightsquigarrow v_2\text{ in } G_{\bar{S}}\big\}\\
&\Phi_5=\big\{v\in \bar{S}: \exists v'\in S, v\rightsquigarrow v'\text{ in } G\big\}\cup S
\end{align*}
, where '$v_1\rightsquigarrow v \text{ in } G_{\bar{S}}$' means there exists a path from $v_1$ to $v$ in $G_{\bar{S}}$; same rule applies to the others.

Then we conclude that $v_1\perp v_2\mid S$  i.f.f. all the following three conditions hold.
\begin{itemize}
\item[1)] $v_2\notin \Phi_1 $ and $v_1\notin \Phi_2$
\item[2)] $\Phi_3\cap \Phi_4=\phi$
\item[3)] either $v_1$ or $v_2$ is disconnected with $\Phi_5$ in $G$
\end{itemize}

The correctness of the above solution is actually quite straightforward. The only thing we should focus on is that we should regard $S$ as a multi-dimensional random variable so that we don't need to worry about the structure inside $S$.

\end{document} 